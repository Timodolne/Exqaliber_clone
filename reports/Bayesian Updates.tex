\documentclass[]{report}


% Title Page
\title{Bayesian Updates}
\usepackage{amsmath,amsthm,amssymb}


\begin{document}
\maketitle
\chapter{Preliminaries}

\section{Circular Distributions}
I'm anticipating that I might need to put more words into this later on, so am leaving space for them here.

\subsection{Von Mises distribution}
The Von-Mises distribution is given by:
\[
f(x, \mu, \kappa) = \frac{1}{2 \pi I_0{\kappa}} \exp(\kappa \cos(x-\mu)), \quad - \pi \le x \le \pi,
\]

where $I_0(\cdot)$ is the $0$th modified Bessel function, where the $n$th modified Bessel function is given by
\[
I_{n}(\kappa) = \frac{1}{\pi} \int_0^\pi \cos(n\theta) \exp(\kappa \cos \theta) \text{d}\theta.
\]

The circular mean of the Von-Mises distribution is given by:
\begin{align*}
	\mathbb{E} [e^{i \theta}] &= \frac{1}{2 \pi I_0(\kappa)} \int_{- \pi}^{\pi} e^{i \theta} \exp(\kappa \cos (\theta - \mu)  \text{d}\theta   \\
	& = \frac{I_{1}(\kappa)}{I_0(\kappa)} e^{i \mu}
\end{align*}
In particular, since $\frac{I_{1}(\kappa)}{I_0(\kappa)}$ is an increasing function, the circular mean uniquely defines a Von-Mises distribution.

For higher circular moments,

\[
\mathbb{E} [e^{i n \theta}] = \frac{I_{|n|}(\kappa)}{I_0(\kappa)} e^{i n \mu}
\]

\chapter{Problem Statement}
\section{Setup}
Goal: Given a single measurement of a Bernoulli random variable and a Von-Mises prior distribution, calculate the posterior distribution and approximate to a Von-Mises distribution.

\begin{itemize}
	\item $t$ - time step
	\item $d_t$ - Grover depth of quantum circuit at time $t$
	\item $Y_t$ - random variable representing a single shot measurement $y_t$ of the quantum circuit at time $t$
	\item $\Pi_(\theta| Y_1 = y_1, \ldots, Y_t = y_t) = \Pi(\theta | \mathbf{Y}_t )$ - 'true' posterior at time $t$ (though values for $t ' < t$ have been used to approximate the earlier distributions)
	\item  $\hat{\Pi}(\theta| Y_1 = y_1, \ldots, Y_t = y_t) = \hat{\Pi}(\theta | \mathbf{Y}_t )$ - approximate posterior at time $t$.
\end{itemize}

According to Bayes rule:
\[
\Pi (\theta | Y_t = y_t, \mathbf{Y}_{t -1}) = \frac{\Pi(Y_t = y_t | \theta) \Pi(\theta | \mathbf{Y}_{t-1})}{\Pi(Y_t = y_t) },
\]
so we need to compute each of the quantities on the RHS.

At time $t$, we make a measurement $y_t$ of $Y_t \sim \text{Ber}(p_t)$ at a Grover depth of $d_t$ where
\[
	p_t = \frac{1}{2}(1 - \cos((4d_t + 2) \hat{\mu}_{t-1}).
\]
Thus,
\[
\Pi(Y_t = y_t | \theta) =  \frac{1}{2}(1 + (-1)^{y_t} \cos((4d_t + 2) \hat{\mu}_{t-1})).
\]
For convenience, let $\lambda_t = 4d_t + 2$.

To simplify some of the computations, we're going to assert that the posterior follows a Von-Mises distribution after every update, so we calculate the new values $\hat{\mu}_t, \hat{\kappa}_t$ and generate our approximate posterior
\[
\hat{\Pi}(\theta | \mathbf{Y}_t )  \sim VM(\hat{\mu}_t, \hat{\kappa}_t).
\]
Now, via Bayes rule, we're computing
\[
\Pi (\theta | Y_t = y_t, \mathbf{Y}_{t -1}) = \frac{\Pi(Y_t = y_t | \theta) \hat{\Pi}(\theta | \mathbf{Y}_{t-1})}{\Pi(Y_t = y_t) },
\]

\section{Single shot updates}
For simplicity, we're going to consider the first step of the update, which makes things a lot nicer. In this case, we want to know what the circular mean of the posterior distribution is after updating.

\begin{itemize}
	\item $\Pi(\theta) \sim VM(\mu, \kappa)$ - prior
	\item $\Pi(Y | \theta) \sim \text{Ber}(\frac{1}{2}(1 - \cos(\lambda \theta)))$
\end{itemize}

This gives us:
\begin{align*}
	\Pi(Y = y) &= \frac{1}{2 \pi I_0(\kappa)} \int_{- \pi}^{\pi} \frac{1}{2} (1 + (-1)^y \cos(\lambda \theta)) \exp(\kappa \cos(\theta - \mu)) \text{d} \theta \\
	&= \frac{1}{2 \pi I_0(\kappa)} \left( \int_{- \pi}^{\pi} \frac{1}{2} \exp(\kappa \cos(\theta - \mu)) \text{d} \theta \right. \\
	 &\left. + (-1)^y \int_{- \pi}^{\pi} \frac{1}{2} \cos(\lambda \theta) \exp(\kappa \cos(\theta - \mu)) \text{d} \theta  \right) \\
	&= \frac{1}{2}\left(1 + (-1)^y \cos(\lambda \mu) \frac{I_\lambda(\kappa)}{I_0(\kappa)}\right)  \\
\end{align*}

Putting this all together, and letting

\[
C(y, \lambda, \kappa) = \frac{\frac{1}{2} \frac{1}{2 \pi I_0(\kappa)}}{\frac{1}{2} (1 + (-1)^y \cos(\lambda \mu) \frac{I_\lambda(\kappa)}{I_0(\kappa)} )} = \frac{1}{ 2 \pi(I_0(\kappa) + (-1)^y \cos(\lambda \mu) I_\lambda(\kappa))}
\]
gives

\begin{align*}
	\mathbb{E}[e^{i \theta} | Y = y] &= C(y, \lambda, \kappa) \int_{- \pi}^{\pi} e^{i \theta} (1 + (-1)^y \cos(\lambda \theta)) \exp(\kappa \cos(\theta - \mu)) \text{d} \theta \\
& =	C(y, \lambda, \kappa) \left(  \int_{- \pi}^{\pi} e^{i \theta} \exp(\kappa \cos(\theta - \mu)) \text{d} \theta \right. \\
 & \quad + (-1)^y  \left. \int_{- \pi}^{\pi} e^{i \theta} \cos(\lambda \theta)   \exp(\kappa \cos(\theta - \mu)) \text{d} \theta \right)\\
 &= C(y, \lambda, \kappa) \left( \vphantom{\int_{-\pi}^\pi} 2 \pi I_1(\kappa) e^{i \mu} \right. \\
 &\quad \left.+ (-1)^y  \int_{-\pi}^{\pi} e^{i \theta} \left( \frac{e^{i \lambda \theta} + e^{- i \lambda \theta}}{2}\right)  \exp(\kappa \cos(\theta - \mu)) \text{d} \theta \right) \\
 &= 2 \pi C(y, \lambda, \kappa) \left( I_1(\kappa) e^{i \mu} + \frac{(-1)^y}{2} \left(I_{\lambda + 1}(\kappa)e^{i(\lambda + 1) \mu}  + I_{\lambda - 1}(\kappa) e^{-i(\lambda + 1)\mu}\right) \right)
\end{align*}
where in the penultimate line, we use the fact that
\[
\int_{- \pi}^ \pi e^{i n \theta} \exp(\kappa \cos(\theta - \mu)) \text{d} \theta = 2 \pi I_0(\kappa) \mathbb{E}[e^{i n \theta}] = I_{|n|}(\kappa) e^{i n \mu}.
\]

This gives us that
\[
\mathbb{E}[e^{i \theta} | Y = y] = \frac{ I_1(\kappa) e^{i \mu} + \frac{(-1)^y}{2} \left(I_{\lambda + 1}(\kappa)e^{i(\lambda + 1) \mu}  + I_{\lambda - 1}(\kappa) e^{-i(\lambda + 1)\mu}\right)}{I_0(\kappa) + (-1)^y \cos(\lambda \mu) I_\lambda(\kappa)}.
\]

If we then take expectations over $Y$ (i.e. multiply by $\Pi(Y = y))$ and sum) this gives us

\[
\mathbb{E}[e^{i \theta}] = \frac{I_{1}(\kappa)}{I_0(\kappa)}e^{i \mu}.
\]
This is problematic. We're not expecting $ \mu$ to move anywhere on average, but we're hoping that $\kappa$ is going to increase ($1/ \kappa$ is analogous to $\sigma^2$ for a normal distribution.



\end{document}
