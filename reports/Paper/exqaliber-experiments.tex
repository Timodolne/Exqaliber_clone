\section{Experimentation}
Describe experiments and performance measures
Two measures of interest, time to solution and quality of solutions

\subsection{Simulation experiments}
Details on setup both with circuits and with direct sampling
\subsubsection{Noiseless experiments}

\subsubsection{Decoherence noise experiments}
Focused on direct sampling case

\subsection{Real hardware experiments}
Include if we decide to carry these out.


%Describe experiments and performance measures
%Two measures of interest, time to solution and quality of solutions

% This is probably explained earlier, but I need it for my argument, so I wrote it. In the final version, we probably need to refer to earlier in the text.
In amplitude estimation, there is an operator $\mathcal{A}$ acting on $n+1$ qubits, such that $\mathcal{A} \ket{\psi}_{n+1} = \sqrt{a}\ket{\psi_1}_{n}\ket{1} + \sqrt{1-a}\ket{\psi_0}_n\ket{0}$,
with $a\in [0,1]$ or, when defining $a=\sin^2{\theta_a}$ with $\theta_a \in [0, \pi]$,
\begin{equation}
	\mathcal{A}_{n+1} \ket{\psi} = \sin{\theta_a} \ket{\psi_1}_n \ket{1}
	+ \cos{\theta_a}\ket{\psi_0}_n\ket{0}.
\end{equation}
The goal is to find $\theta_a$.

One can create an operator $\mathcal{Q} = \mathcal{A} \mathcal{S}_0 \mathcal{A}^{\dagger}\mathcal{S}_{\psi_0}$ with $\mathcal{S}_{\psi_0} = \mathbb{I} - 2\ket{\psi_0}_n\bra{\psi_0}_n\otimes\ket{0}\bra{0}$ and $\mathcal{S}_0 = \mathbb{I} - 2\ket{0}_{n+1}\bra{0}_{n+1}$. % TODO reference
% TODO choose definition of 'oracle queries'. In the iterative AE paper, they "denote applications of Q as quantum samples or oracle queries."
Now by applying $\mathcal{Q}$ for $k$ times (and thereby applying $\mathcal{A}$ a total of $(2k+1)$ times), the probability of measuring the $\ket{1}$ state follows a Bernoulli distribution with $p$ equal to
\begin{equation}
	p = \frac{1}{2}(1-\cos{((2k+1)\theta_a)}).
	\label{eq:bernoulli-p}
\end{equation}
Using this equation, one can test different amplitude estimation analytically, by sampling from the Bernoulli distribution directly. We call this direct analytical sampling.

We experimented with our novel amplitude estimation routine. There are two relevant measures for amplitude estimation.
These are the time to solution (and its scaling) and the quality of the solution, i.e., the error $\epsilon$.
To measure these, we simulate experiments using direct analytical sampling.


As our algorithm uses dynamical updates, depending on the true $\theta$, %TODO: make this consistent throughout the article. Maybe theta hat?
anomalies can rise for values around integer fractions of $\pi$.
