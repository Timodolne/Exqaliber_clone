\documentclass[a4paper,onecolumn,11pt,unpublished]{quantumarticle}
\pdfoutput=1
\usepackage[utf8]{inputenc}
\usepackage[english]{babel}
\usepackage[T1]{fontenc}
\usepackage{amsmath}
\usepackage{amssymb}
\usepackage{hyperref}

\makeatletter
\newcommand{\bra}[1]{
  \langle #1|
}
\newcommand{\ket}[1]{
  |#1\rangle
}
\newcommand{\braket}[2]{
  \langle #1|#2\rangle
}
\makeatother

\usepackage{tikz}
\usepackage{lipsum}

\newtheorem{theorem}{Theorem}

\begin{document}

\title{Exqaliber Stage 1 Report}

\author{James Cruise}
\affiliation{Cambridge Consultants, Cambridge, UK}
\orcid{0000-0002-8127-1528}
\email{james.cruise@cambridgeconsultants.com}
\author{Joseph Tedds}
\email{joseph.tedds@cambridgeconsultants.com}
\affiliation{Cambridge Consultants, Cambridge, UK}
\author{Camille de Valk}
\email{camille.de.valk@capgemini.com}
\affiliation{Capgemini QLab}
\maketitle

\begin{abstract}
 TBD
\end{abstract}


\section{Introduction}
Basic introduction giving definition and contribution

\subsection{Amplitude estimation}
Details of the amplitude estimation and background on its use, including Quantum Monte Carlo

\subsubsection{Statistical amplitude estimation}
Introduce idea of statistical learning of parameter to reduce circuit length

\subsubsection{Decoherent noise model}
Introduce model for decoherent noise and associated effects on statistical problem

\subsection{Related work}
Review of related work and papers

\subsubsection{Quantum phase estimation}
I am not sure if we want to make the connection in a public document but it is useful to make in an internal report

\subsection{Contribution}
This should be written last and include a map for the rest of the paper


\section{Algorithm}

\subsection{Algorithm description}
Give a statement of our algorithm, phrase as an algorithm rather than talking about Bayes updates etc. What is implemented

State is current estimate and variance
Rules for updating state given result
Rules for selecting next update

Include detail of stopping criteria

\subsubsection{Adjustments for noise}
Changes needed for decoherence noise

\subsection{Bayesian framework}
Introduction to Bayesian stats, prior, posterior, and adaptive termination

Introduce idea of using normal distribution


\subsubsection{Posterior properties}
Posterior mean, variance and expected variance under noiseless and decoherence noise model

\subsubsection{Normal approximation of posterior}
Details on the use of a normal approximation for posterior distribution and next prior.
Include comments about both the circular issue and tails due to indeterminacy.

\subsection{Selecting next sample depth}
Details of greedy choice (possibly give indication of dynamic programming approach)
Noise and noiseless

\section{Experimentation}
%Describe experiments and performance measures
%Two measures of interest, time to solution and quality of solutions

% This is probably explained earlier, but I need it for my argument, so I wrote it. In the final version, we probably need to refer to earlier in the text.
In amplitude estimation, there is an operator $\mathcal{A}$ acting on $n+1$ qubits, such that $\mathcal{A} \ket{\psi}_{n+1} = \sqrt{a}\ket{\psi_1}_{n}\ket{1} + \sqrt{1-a}\ket{\psi_0}_n\ket{0}$,
with $a\in [0,1]$ or, when defining $a=\sin^2{\theta_a}$ with $\theta_a \in [0, \pi]$,
\begin{equation}
	\mathcal{A}_{n+1} \ket{\psi} = \sin{\theta_a} \ket{\psi_1}_n \ket{1}
	+ \cos{\theta_a}\ket{\psi_0}_n\ket{0}.
\end{equation}
The goal is to find $\theta_a$.

One can create an operator $\mathcal{Q} = \mathcal{A} \mathcal{S}_0 \mathcal{A}^{\dagger}\mathcal{S}_{\psi_0}$ with $\mathcal{S}_{\psi_0} = \mathbb{I} - 2\ket{\psi_0}_n\bra{\psi_0}_n\otimes\ket{0}\bra{0}$ and $\mathcal{S}_0 = \mathbb{I} - 2\ket{0}_{n+1}\bra{0}_{n+1}$. % TODO reference
% TODO choose definition of 'oracle queries'. In the iterative AE paper, they "denote applications of Q as quantum samples or oracle queries."
Now by applying $\mathcal{Q}$ for $k$ times (and thereby applying $\mathcal{A}$ a total of $(2k+1)$ times), the probability of measuring the $\ket{1}$ state follows a Bernoulli distribution with $p$ equal to
\begin{equation}
	p = \frac{1}{2}(1-\cos{((2k+1)\theta_a)}).
	\label{eq:bernoulli-p}
\end{equation}
Using this equation, one can test different amplitude estimation analytically, by sampling from the Bernoulli distribution directly. We call this direct analytical sampling.

We experimented with our novel amplitude estimation routine. There are two relevant measures for amplitude estimation.
These are the time to solution (and its scaling) and the quality of the solution, i.e., the error $\epsilon$.
To measure these, we simulate experiments using direct analytical sampling.


As our algorithm uses dynamical updates, depending on the true $\theta$, %TODO: make this consistent throughout the article. Maybe theta hat?
anomalies can rise for values around integer fractions of $\pi$.

\subsection{Simulation experiments}
Details on setup both with circuits and with direct sampling



\subsubsection{Noiseless experiments}

\subsubsection{Decoherence noise experiments}
Focused on direct sampling case

\subsection{Real hardware experiments}
Include if we decide to carry these out.


\section{Discussion}
This is for internal discussions not external release
Topics to include here include:
\begin{itemize}
	\item Directions for future including reinforcement learning
	\item Superconducting and trapped ion thought experiment
	\item Discussion about general dynamic programming approach for considering error mitigation and correction
	\item Connection to QPE and applicability of thinking to that setting (maybe include small section on the more complicated statistical challenge of general state)
	\item Circular distributions and what we know about them (probably subsection of this section or algorithm)
\end{itemize}


\bibliographystyle{quantum}
\begin{thebibliography}{9}


\bibitem{jj}
gg
\end{thebibliography}




\appendix

\section{Technical details}
Details of technical work if needed
\end{document}
