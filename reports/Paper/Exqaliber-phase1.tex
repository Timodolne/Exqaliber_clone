\documentclass[a4paper,onecolumn,11pt,unpublished]{quantumarticle}
\pdfoutput=1
\usepackage[utf8]{inputenc}
\usepackage[english]{babel}
\usepackage[T1]{fontenc}
\usepackage{amsmath}
\usepackage{hyperref}

\usepackage{tikz}
\usepackage{lipsum}

\newtheorem{theorem}{Theorem}

\begin{document}

\title{Exqaliber Stage 1 Report}

\author{James Cruise}
\affiliation{Cambridge Consultants, Cambridge, UK}
\orcid{0000-0002-8127-1528}
\email{james.cruise@cambridgeconsultants.com}
\author{Joseph Tedds}
\email{joseph.tedds@cambridgeconsultants.com}
\affiliation{Cambridge Consultants, Cambridge, UK}
\author{Camille de Valk}
\email{camille.de.valk@capgemini.com}
\affiliation{Capgemini QLab}
\maketitle

\begin{abstract}
 TBD
\end{abstract}


\section{Introduction}
Basic introduction giving definition and contribution

\subsection{Amplitude estimation}
Details of the amplitude estimation and background on its use, including Quantum Monte Carlo

\subsubsection{Statistical amplitude estimation}
Introduce idea of statistical learning of parameter to reduce circuit length

\subsubsection{Decoherent noise model}
Introduce model for decoherent noise and associated effects on statistical problem

\subsection{Related work}
Review of related work and papers

\subsubsection{Quantum phase estimation}
I am not sure if we want to make the connection in a public document but it is useful to make in an internal report

\subsection{Contribution}
This should be written last and include a map for the rest of the paper


\section{Algorithm}

\subsection{Algorithm description}
Give a statement of our algorithm, phrase as an algorithm rather than talking about Bayes updates etc. What is implemented

State is current estimate and variance
Rules for updating state given result
Rules for selecting next update

Include detail of stopping criteria

\subsubsection{Adjustments for noise}
Changes needed for decoherence noise

\subsection{Bayesian framework}
Introduction to Bayesian stats, prior, posterior, and adaptive termination

Introduce idea of using normal distribution 


\subsubsection{Posterior properties}
Posterior mean, variance and expected variance under noiseless and decoherence noise model

\subsubsection{Normal approximation of posterior}
Details on the use of a normal approximation for posterior distribution and next prior.
Include comments about both the circular issue and tails due to indeterminacy. 

\subsection{Selecting next sample depth}
Details of greedy choice (possibly give indication of dynamic programming approach)
Noise and noiseless

\section{Experimentation}
Describe experiments and performance measures
Two measures of interest, time to solution and quality of solutions

\subsection{Simulation experiments}
Details on setup both with circuits and with direct sampling
\subsubsection{Noiseless experiments}
	
\subsubsection{Decoherence noise experiments}
Focused on direct sampling case

\subsection{Real hardware experiments}
Include if we decide to carry these out. 
	

\section{Discussion}
This is for internal discussions not external release
Topics to include here include:
\begin{itemize}
	\item Directions for future including reinforcement learning
	\item Superconducting and trapped ion thought experiment
	\item Discussion about general dynamic programming approach for considering error mitigation and correction
	\item Connection to QPE and applicability of thinking to that setting (maybe include small section on the more complicated statistical challenge of general state)
	\item Circular distributions and what we know about them (probably subsection of this section or algorithm)
\end{itemize}


\bibliographystyle{quantum}
\begin{thebibliography}{9}


\bibitem{jj}
gg
\end{thebibliography}




\appendix

\section{Technical details}
Details of technical work if needed
\end{document}
